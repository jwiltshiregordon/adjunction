\documentclass[tikz]{standalone}
%\usetikzlibrary{calc}
\usepackage{tikz-cd}
\usepackage{amsmath}
\usepackage{pict2e}
\usepackage{amsmath,amsthm,amssymb,graphicx,mathtools,tikz,hyperref}
\newcommand{\amsamp}{&}
\begingroup
\catcode`\&=13
\gdef\pampmatrix{%
  \begingroup
  \let&=\amsamp
  \begin{bmatrix}%
}
\gdef\endpampmatrix{\end{bmatrix}\endgroup}
\endgroup
\setcounter{MaxMatrixCols}{75}
\begin{document}
\begin{tikzcd}E^{pq}_1 \quad p = &0 &[26em]1 &[26em]2\\
q = 1: & \mathbb{Z}^{6} \arrow[r, "\begin{bmatrix} 1 \amsamp 0 \amsamp 0 \amsamp 0 \amsamp 0 \amsamp 0 \amsamp 0 \amsamp 0 \amsamp 0 \amsamp 0 \amsamp 0 \amsamp 0 \amsamp 0 \amsamp 0 \amsamp 0 \\ 0 \amsamp 1 \amsamp 0 \amsamp 0 \amsamp 0 \amsamp 0 \amsamp 0 \amsamp 0 \amsamp 0 \amsamp 0 \amsamp 0 \amsamp 0 \amsamp 0 \amsamp 0 \amsamp 0 \\ 0 \amsamp 0 \amsamp 1 \amsamp 0 \amsamp 0 \amsamp 0 \amsamp 0 \amsamp 0 \amsamp 0 \amsamp 0 \amsamp 0 \amsamp 0 \amsamp 0 \amsamp 0 \amsamp 0 \\ 0 \amsamp 0 \amsamp 0 \amsamp 1 \amsamp 0 \amsamp 0 \amsamp 0 \amsamp 0 \amsamp 0 \amsamp 0 \amsamp 0 \amsamp 0 \amsamp 0 \amsamp 0 \amsamp 0 \\ 0 \amsamp 0 \amsamp 0 \amsamp 0 \amsamp 1 \amsamp 0 \amsamp 0 \amsamp 0 \amsamp 0 \amsamp 0 \amsamp 0 \amsamp 0 \amsamp 0 \amsamp 0 \amsamp 0 \\ -1 \amsamp -1 \amsamp 1 \amsamp -1 \amsamp -1 \amsamp 2 \amsamp 0 \amsamp 0 \amsamp 0 \amsamp 0 \amsamp 0 \amsamp 0 \amsamp 0 \amsamp 0 \amsamp 0\end{bmatrix}" description] & \mathbb{Z}^{15} \arrow[r, "\begin{bmatrix} 0 \amsamp 0 \amsamp 0 \amsamp 0 \amsamp 0 \amsamp 0 \amsamp 0 \amsamp 0 \amsamp 0 \amsamp 0 \\ 0 \amsamp 0 \amsamp 0 \amsamp 0 \amsamp 0 \amsamp 0 \amsamp 0 \amsamp 0 \amsamp 0 \amsamp 0 \\ 0 \amsamp 0 \amsamp 0 \amsamp 0 \amsamp 0 \amsamp 0 \amsamp 0 \amsamp 0 \amsamp 0 \amsamp 0 \\ 0 \amsamp 0 \amsamp 0 \amsamp 0 \amsamp 0 \amsamp 0 \amsamp 0 \amsamp 0 \amsamp 0 \amsamp 0 \\ 0 \amsamp 0 \amsamp 0 \amsamp 0 \amsamp 0 \amsamp 0 \amsamp 0 \amsamp 0 \amsamp 0 \amsamp 0 \\ 0 \amsamp 0 \amsamp 0 \amsamp 0 \amsamp 0 \amsamp 0 \amsamp 0 \amsamp 0 \amsamp 0 \amsamp 0 \\ 1 \amsamp 0 \amsamp 0 \amsamp 0 \amsamp 0 \amsamp 0 \amsamp 0 \amsamp 0 \amsamp 0 \amsamp 0 \\ 0 \amsamp 1 \amsamp 0 \amsamp 0 \amsamp 0 \amsamp 0 \amsamp 0 \amsamp 0 \amsamp 0 \amsamp 0 \\ 0 \amsamp 0 \amsamp 1 \amsamp 0 \amsamp 0 \amsamp 0 \amsamp 0 \amsamp 0 \amsamp 0 \amsamp 0 \\ 0 \amsamp 0 \amsamp 0 \amsamp 1 \amsamp 0 \amsamp 0 \amsamp 0 \amsamp 0 \amsamp 0 \amsamp 0 \\ 0 \amsamp 0 \amsamp 0 \amsamp 0 \amsamp 1 \amsamp 0 \amsamp 0 \amsamp 0 \amsamp 0 \amsamp 0 \\ 0 \amsamp 0 \amsamp 0 \amsamp 0 \amsamp 0 \amsamp 1 \amsamp 0 \amsamp 0 \amsamp 0 \amsamp 0 \\ 0 \amsamp 0 \amsamp 0 \amsamp 0 \amsamp 0 \amsamp 0 \amsamp 1 \amsamp 0 \amsamp 0 \amsamp 0 \\ 0 \amsamp 0 \amsamp 0 \amsamp 0 \amsamp 0 \amsamp 0 \amsamp 0 \amsamp 1 \amsamp 0 \amsamp 0 \\ 0 \amsamp 0 \amsamp 0 \amsamp 0 \amsamp 0 \amsamp 0 \amsamp 0 \amsamp 0 \amsamp 1 \amsamp 0\end{bmatrix}" description] & \mathbb{Z}^{10} & \\[26em] 
 q = 0: & \mathbb{Z}^{6} \arrow[r, "\begin{bmatrix} 1 \amsamp 0 \amsamp 0 \amsamp 0 \amsamp 0 \amsamp 0 \amsamp 0 \amsamp 0 \amsamp 0 \amsamp 0 \amsamp 0 \amsamp 0 \amsamp 0 \amsamp 0 \amsamp 0 \\ 0 \amsamp 1 \amsamp 0 \amsamp 0 \amsamp 0 \amsamp 0 \amsamp 0 \amsamp 0 \amsamp 0 \amsamp 0 \amsamp 0 \amsamp 0 \amsamp 0 \amsamp 0 \amsamp 0 \\ 0 \amsamp 0 \amsamp 1 \amsamp 0 \amsamp 0 \amsamp 0 \amsamp 0 \amsamp 0 \amsamp 0 \amsamp 0 \amsamp 0 \amsamp 0 \amsamp 0 \amsamp 0 \amsamp 0 \\ 0 \amsamp 0 \amsamp 0 \amsamp 1 \amsamp 0 \amsamp 0 \amsamp 0 \amsamp 0 \amsamp 0 \amsamp 0 \amsamp 0 \amsamp 0 \amsamp 0 \amsamp 0 \amsamp 0 \\ 0 \amsamp 0 \amsamp 0 \amsamp 0 \amsamp 1 \amsamp 0 \amsamp 0 \amsamp 0 \amsamp 0 \amsamp 0 \amsamp 0 \amsamp 0 \amsamp 0 \amsamp 0 \amsamp 0 \\ -1 \amsamp -1 \amsamp -1 \amsamp -1 \amsamp -1 \amsamp 0 \amsamp 0 \amsamp 0 \amsamp 0 \amsamp 0 \amsamp 0 \amsamp 0 \amsamp 0 \amsamp 0 \amsamp 0\end{bmatrix}" description] & \mathbb{Z}^{15} \arrow[r, "\begin{bmatrix} 0 \amsamp 0 \amsamp 0 \amsamp 0 \amsamp 0 \amsamp 0 \amsamp 0 \amsamp 0 \amsamp 0 \amsamp 0 \\ 0 \amsamp 0 \amsamp 0 \amsamp 0 \amsamp 0 \amsamp 0 \amsamp 0 \amsamp 0 \amsamp 0 \amsamp 0 \\ 0 \amsamp 0 \amsamp 0 \amsamp 0 \amsamp 0 \amsamp 0 \amsamp 0 \amsamp 0 \amsamp 0 \amsamp 0 \\ 0 \amsamp 0 \amsamp 0 \amsamp 0 \amsamp 0 \amsamp 0 \amsamp 0 \amsamp 0 \amsamp 0 \amsamp 0 \\ 0 \amsamp 0 \amsamp 0 \amsamp 0 \amsamp 0 \amsamp 0 \amsamp 0 \amsamp 0 \amsamp 0 \amsamp 0 \\ 0 \amsamp 0 \amsamp 1 \amsamp 0 \amsamp 0 \amsamp 0 \amsamp 0 \amsamp 0 \amsamp 0 \amsamp 0 \\ 1 \amsamp -2 \amsamp 0 \amsamp 0 \amsamp 0 \amsamp 0 \amsamp 0 \amsamp 0 \amsamp 0 \amsamp 0 \\ 0 \amsamp 0 \amsamp 0 \amsamp 1 \amsamp 0 \amsamp 0 \amsamp 0 \amsamp 0 \amsamp 0 \amsamp 0 \\ 0 \amsamp 0 \amsamp 0 \amsamp 0 \amsamp 1 \amsamp 0 \amsamp 0 \amsamp 0 \amsamp 0 \amsamp 0 \\ 0 \amsamp 0 \amsamp 0 \amsamp 0 \amsamp 0 \amsamp 1 \amsamp 0 \amsamp 0 \amsamp 0 \amsamp 0 \\ 0 \amsamp 0 \amsamp 0 \amsamp 0 \amsamp 0 \amsamp 0 \amsamp 1 \amsamp 0 \amsamp 0 \amsamp 0 \\ 0 \amsamp 0 \amsamp 0 \amsamp 0 \amsamp 0 \amsamp 0 \amsamp 0 \amsamp 1 \amsamp 0 \amsamp 0 \\ 0 \amsamp 0 \amsamp 0 \amsamp 0 \amsamp 0 \amsamp 0 \amsamp 0 \amsamp 0 \amsamp 1 \amsamp 0 \\ 0 \amsamp 0 \amsamp 0 \amsamp 0 \amsamp 0 \amsamp 0 \amsamp 0 \amsamp 0 \amsamp 0 \amsamp 1 \\ 1 \amsamp 0 \amsamp 0 \amsamp 1 \amsamp 0 \amsamp -1 \amsamp 0 \amsamp -1 \amsamp 0 \amsamp 0\end{bmatrix}" description] & \mathbb{Z}^{10} & \\[26em] \hline
 \end{tikzcd}
 \begin{tikzcd}
 E^{pq}_2 \quad p = &0 &[26em]1 &[26em]2 \\
 q = 1: & \; & \mathbb{Z}/2 & \mathbb{Z} & \\[26em] 
 q = 0: & \mathbb{Z} & \; & \mathbb{Z}/2 & \\[26em] \hline
 \end{tikzcd}
 \begin{tikzcd}
 E^{pq}_3 \quad p = &0 &[26em]1 &[26em]2 \\
 q = 1: & \; & \mathbb{Z}/2 & \mathbb{Z} & \\[26em] 
 q = 0: & \mathbb{Z} & \; & \mathbb{Z}/2 & \\[26em] \hline
 \end{tikzcd}
 \begin{tikzcd}
 E^{pq}_4 \quad p = &0 &[26em]1 &[26em]2 \\
 q = 1: & \; & \mathbb{Z}/2 & \mathbb{Z} & \\[26em] 
 q = 0: & \mathbb{Z} & \; & \mathbb{Z}/2 & \\[26em] \hline
 \end{tikzcd}
 \begin{tikzcd}
 E^{pq}_5 \quad p = &0 &[26em]1 &[26em]2 \\
 q = 1: & \; & \mathbb{Z}/2 & \mathbb{Z} & \\[26em] 
 q = 0: & \mathbb{Z} & \; & \mathbb{Z}/2 & \\[26em] \hline
 \end{tikzcd}
 \begin{tikzcd}
 E^{pq}_6 \quad p = &0 &[26em]1 &[26em]2 \\
 q = 1: & \; & \mathbb{Z}/2 & \mathbb{Z} & \\[26em] 
 q = 0: & \mathbb{Z} & \; & \mathbb{Z}/2 & \\[26em] \hline
 \end{tikzcd}
 \begin{tikzcd}
 \end{tikzcd}
% Further ?tikzpicture? environments are possible which will create further pages.
\end{document}